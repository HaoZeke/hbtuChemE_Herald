%%%%%%%%%%%%%%%%%%%%%%%%%%%%%%%%%%%%%%%%%
% Professional Newsletter Template
% LaTeX Template
% Version 1.0 (09/03/14)
%
% Created by:
% Bob Kerstetter (https://www.tug.org/texshowcase/) and extensively modified by:
% Vel (vel@latextemplates.com)
% 
% This template has been downloaded from:
% http://www.LaTeXTemplates.com
%
% License:
% CC BY-NC-SA 3.0 (http://creativecommons.org/licenses/by-nc-sa/3.0/)
%
%%%%%%%%%%%%%%%%%%%%%%%%%%%%%%%%%%%%%%%%%

\documentclass[11pt]{article} % The default font size is 10pt; 11pt and 12pt are alternatives

%%%%%%%%%%%%%%%%%%%%%%%%%%%%%%%%%%%%%%%%%
% Professional Newsletter Template
% Structural Definitions File
% Version 1.0 (09/03/14)
%
% Created by:
% Vel (vel@latextemplates.com)
% 
% This file has been downloaded from:
% http://www.LaTeXTemplates.com
%
% License:
% CC BY-NC-SA 3.0 (http://creativecommons.org/licenses/by-nc-sa/3.0/)
%
%%%%%%%%%%%%%%%%%%%%%%%%%%%%%%%%%%%%%%%%%

%----------------------------------------------------------------------------------------
%	REQUIRED PACKAGES
%----------------------------------------------------------------------------------------

\usepackage{graphicx} % Required for including images
\usepackage{microtype} % Improved typography
\usepackage{multicol} % Used for the two-column layout of the document
\usepackage{booktabs} % Required for nice horizontal rules in tables
\usepackage{wrapfig} % Required for in-line images
\usepackage{float} % Required for forcing figures not to float with the [H] parameter
\usepackage{fancyhdr} % For better header footer setup

% -----------------------------------------------
% References

\usepackage[
backend=biber,
autocite = superscript,
bibstyle=phys,
backref=true,
hyperref=true,
block=none,
natbib=true,
doi=true,
uniquename=false,
autopunct=true,
sorting=nyt
]{biblatex}

\makeatletter
% Citation Fine Tuning
\DeclareFieldFormat{edition}        % Changes spacing for the edition numbers.
                   {\ifinteger{#1}
                    {\mkbibordedition{#1}\addthinspace{}ed.}
                    {#1\isdot}}
% Prints author names as small caps
\renewcommand{\mkbibnamegiven}[1]{\textsc{#1}}
\renewcommand{\mkbibnamefamily}[1]{\textsc{#1}}
\renewcommand{\mkbibnameprefix}[1]{\textsc{#1}}
\renewcommand{\mkbibnamesuffix}[1]{\textsc{#1}}
\makeatother

% Add Bibs
\addbibresource{refs.bib}

%------------------------------------------------
% Fonts

\usepackage{charter} % Use the Charter font as the main document font
\usepackage{courier} % Use the Courier font for \texttt (monospaced) only
\usepackage[T1]{fontenc} % Use T1 font encoding
% \usepackage{tgbonum}
% \usepackage{tgpagella}
% \usepackage{tgschola}
% \usepackage{fourier}

%------------------------------------------------
% List Separation

\usepackage{enumitem} % Required to customize the list environments
\setlist{noitemsep,nolistsep} % Remove spacing before, after and within lists for a compact look

%------------------------------------------------
% Figure and Table Caption Styles

\usepackage{caption} % Required for changing caption styles
\captionsetup[table]{labelfont={bf,sf},labelsep=period,justification=justified} % Specify the table caption style
\captionsetup[figure]{labelfont={sf,bf},labelsep=period,justification=justified, font=small} % Specify the figure caption style
\setlength{\abovecaptionskip}{10pt} % Whitespace above captions

%------------------------------------------------
% Spacing Between Paragraphs

\makeatletter
\usepackage{parskip}
\setlength{\parskip}{6pt}
\newcommand{\@minipagerestore}{\setlength{\parskip}{6pt}}
\makeatother

%----------------------------------------------------------------------------------------
%	PAGE MARGINS AND SPACINGS
%----------------------------------------------------------------------------------------

\textwidth = 7 in % Text width
\textheight = 10 in % Text height
\oddsidemargin = -18pt % Left side margin on odd pages
\evensidemargin = -18pt % Left side margin on even pages
\topmargin = -36pt % Top margin
\headheight = 0pt % Remove the header by setting its space to 0
\headsep = 0pt % Remove the space between the header and top of the page
\parskip = 4pt % Space between paragraph
\parindent = 0.0in % Paragraph indentation
\pagestyle{empty} % Disable page numbering

% Add to the above with fancyhdr
% \renewcommand{\headrulewidth}{0pt}
% \renewcommand{\footrulewidth}{0pt}
% \setlength\headheight{80.0pt}
% \addtolength{\textheight}{-80.0pt}

% This is the logo location now
% \chead{\includegraphics[width=0.15\linewidth]{../img/logo_square.png}}

% Now put the header on page one
% \thispagestyle{fancy}

%----------------------------------------------------------------------------------------
%	COLORS
%----------------------------------------------------------------------------------------

\usepackage[dvipsnames,svgnames]{xcolor} % Required to specify custom colors

\definecolor{altncolor}{rgb}{.8,0,0} % Dark red
%\definecolor{altncolor}{rgb}{.2,.4,.8} % Dark blue
%\definecolor{altncolor}{rgb}{.84,.16,.16} % Red

\usepackage[colorlinks=true, linkcolor=altncolor, anchorcolor=altncolor, citecolor=altncolor, filecolor=altncolor, menucolor=altncolor, urlcolor=altncolor]{hyperref} % Use the color defined above for all links

%----------------------------------------------------------------------------------------
%	BOX STYLES
%----------------------------------------------------------------------------------------

\usepackage[framemethod=TikZ]{mdframed}% Required for creating boxes
\mdfdefinestyle{sidebar}{
    linecolor=black, % Outer line color
    outerlinewidth=0.5pt, % Outer line width
    roundcorner=0pt, % Amount of corner rounding
    innertopmargin=10pt, % Top margin
    innerbottommargin=10pt, % Bottom margin
    innerrightmargin=10pt, % Right margin
    innerleftmargin=10pt, % Left margin
    backgroundcolor=white, % Box background color
    frametitlebackgroundcolor=white, % Title background color
    frametitlerule=false, % Title rule - true or false
    frametitlerulecolor=white, % Title rule color
    frametitlerulewidth=0.5pt, % Title rule width
    frametitlefont=\Large, % Title heading font specification
    font=\small
}

\mdfdefinestyle{intextbox}{
    linecolor=black, % Outer line color
    outerlinewidth=0.5pt, % Outer line width
    roundcorner=10pt, % Amount of corner rounding
    innertopmargin=7pt, % Top margin
    innerbottommargin=7pt, % Bottom margin
    innerrightmargin=7pt, % Right margin
    innerleftmargin=7pt, % Left margin
    backgroundcolor=white, % Box background color
    frametitlebackgroundcolor=white, % Title background color
    frametitlerule=false, % Title rule - true or false
    frametitlerulecolor=white, % Title rule color
    frametitlerulewidth=0.5pt, % Title rule width
    frametitlefont=\Large % Title heading font specification
}

%----------------------------------------------------------------------------------------
%	HEADING STYLE
%----------------------------------------------------------------------------------------

\newcommand{\heading}[2]{ % Define the \heading command
\vspace{#2} % White space above the heading
{\begin{center}\Large\textbf{#1}\end{center}} % The heading style
\vspace{#2} % White space below the heading
}

\newcommand{\BackToContents}{\hyperlink{contents}{{\small Back to Contents}}} % Define a command for linking back to the contents of the newsletter


\newcommand\deco[2]{%
  \par\vspace{1ex}
  \begin{center}
  \fontsize{#1}{#1}\usefont{U}{webo}{xl}{n}#2
  \end{center}
  \vspace*{1ex}\par
}
 % Include the document which specifies all packages and structural customizations for this template

\begin{document}

% ----------------------------------------------------------------------------------------
% HEADER IMAGE
% ----------------------------------------------------------------------------------------

\begin{figure}[H]
  \centering\includegraphics[width=0.2\linewidth, keepaspectratio]{../img/logo_square.png}
\end{figure}
% ----------------------------------------------------------------------------------------
% SIDEBAR - FIRST PAGE
% ----------------------------------------------------------------------------------------
\begin{minipage}[t]{.30\linewidth} % Mini page taking up 30% of the actual page
  \begin{mdframed}[style=sidebar,frametitle={}] % Sidebar box

    % -----------------------------------------------------------

    \hypertarget{contents}{\textbf{{\large In This
          Issue\ldots}}} % \hypertarget provides a label to reference using \hyperlink{label}{link text}
    \begin{itemize}
      \setlength\itemsep{1em}
    \item \hyperlink{hodOffice}{From the HOD's office} % These link to their appropriate sections in the newsletter
    \item \hyperlink{moneyHungry}{Privatization of educational software}
    \item \hyperlink{fireBalls}{Environment friendly fire extinguishers}
    \item \hyperlink{desalination}{Water harvesting via desalination}
    \end{itemize}

    \centerline {\rule{.75\linewidth}{.25pt}} % Horizontal line
    
  \end{mdframed}\hfill

  % ----------------------------------------------------------------------------------------

  \centering
  \begin{minipage}[t]{.95\linewidth}
    \textbf{Mail submissions, suggestions etc. to: \\}
    \textit{hbtucheme@gmail.com \\~\\}    
    \textbf{Editor:}\\
    \small Rohit Goswami \\
    Final Chemical
  \end{minipage}

  % ----------------------------------------------------------------------------------------

\end{minipage}\hfill % End the sidebar mini page
% ----------------------------------------------------------------------------------------
% IN-TEXT BOX
% ----------------------------------------------------------------------------------------
\begin{minipage}[t]{.66\linewidth}

\begin{mdframed}[style=intextbox,frametitle={}] % Sidebar box

  \hypertarget{hodOffice}{\heading{From the Head's Office}{0pt}} % \hypertarget provides a label to reference using \hyperlink{label}{link text}
  It is with great pleasure that I present this inaugral issue of the Chemical
  Engineering Herald (ChemE Herald). This technical periodical newsletter is a
  departmental initiative to showcase and foster the growth and exposition of
  technical expertise. \\
  --- \textsc{\textbf{Prof. S. K. Gupta}}
\end{mdframed}

  \hypertarget{moneyHungry}{\heading{Privatization of educational
      software}{6pt}}  % \hypertarget provides a label to reference using \hyperlink{label}{link text}
  { \small  \textbf{By: Rohit Goswami, Final ChemE}\\ }
  From process simulation techniques to NIST (SUPERTAPP) databases for chemical engineering
  data, there has been an inexorable march towards monetization and
  privatization of educational software. This is damning for aspiring chemical
  engineering students, as the software on which real predictions are based are
  being efficiently and quickly subsumed into various commercial software
  systems, like ASPEN and others. Even
  \autocite{towlerChemicalEngineeringDesign2013} states emphatically that the
  latest cost correlations and affiliated costing data are available via ASPEN
  only, and has not been published in open literature.

  For a subject based on the sharing of industrial knowledge, with roots in the
  ACS (American Chemical Society) this is a grave state of affairs and is a
  strange irony. The most striking example of this wealth induced decline of academic
  resources is of course that of the erstwhile CHESS (Chemical Engineering
  Simulation System), developed by Dr. Rudy Motard and Dr. Ernest Henley at the
  University of Houston with a grant from the U.S. Navy. The developers of CHESS
  after gaining much academic merit chose to sell their rights to the software
  which has now become unavailable to students and is now marketed as CHEMCAD.

  For the basic sciences there are still databases of freely available, robustly
  verified data, however, lest this miserable state of decline be noted and
  halted, our subject will decline away into nothingness. At this rate, the
  efforts of stalwarts such as Dryden, Sherwood, McCabe et. al who chose to
  disseminate their findings for the education of future generations will go in
  vain as there are none who carry forth their legacy.

  This is not meant to be an invective piece on the relative merits of paid and
  open source software, however it must be noted with praise that the computer
  science discipline has never faced a slump by open sourcing their data and
  software tools. No company keeps secret in house programming languages,
  whereas in our industries, secrecy and stringent patent laws are the norm.
\end{minipage}
%
% ----------------------------------------------------------------------------------------
% MAIN BODY 
% ----------------------------------------------------------------------------------------
%

\deco{10pt}{$\clubsuit$~$\clubsuit$~$\clubsuit$}
  \hypertarget{fireBalls}{\heading{Environment friendly fire extinguishers}{6pt}}  % \hypertarget provides a label to reference using \hyperlink{label}{link text}
  { \small  \textbf{By: Bhawani Shankar Tiwari, Third ChemE}\\ }

\begin{multicols}{2} % Two-column layout
The aim behind the development of new technology is for the production of novel,
halogen-free, environmentally safe, highly efficient fire-extinguishing powders.
It is based on local mineral raw materials and elaboration of new types of
fire-protective materials. The basis of such fire-extinguishing powders are the
composite materials which function as efficient flame retardant.

New technological developments are done which comprise of Micro Fog System and
Low Frequency Sound which are more efficient and eco-safe than the traditional
fire extinguishers. Fire extinguishing powders are prepared by grinding the raw
materials followed by screening up to 250 micrometer dispersity, drying at
700-1000 degree Celsius and mixing of raw materials.

It does not require any additional chemical processing and modifications like
expensive halogen containing hydrophobizing additives. Raw materials zeolite,
clay shale and perlite are chosen due to their high performance properties and
the ability to suppress the combustion and burning processes. They are of
silicate origin, containing alkali and alkaline-earth metal carbonates,
bicarbonates and iron, aluminium, alkali metal hydroxides. Therefore, at high
temperatures these raw materials are characterized with emission of
incombustible gases, steam and metal oxides which dilute combustible products,
creating protective film and coke layer on the surface of material.

On the basis of thermogravimetric analysis it is stated that at the first stage
adsorption and crystallization water separation takes place. At the next stage
of higher temperature (700 degree celcius and more) there happens to be the
formation of metal oxides protective film and coke layer. The liberated
incombustible gases and water steam in the flame zone functions as phlegmatizer
which leads to formation of swelled layer in surface zone. The protective film
of metal oxides and coke layer cause a strong fire-limiting effect. This
indicates the fact that these materials can exhibit the properties similar to
highly effective homogeneous inhibitors.

\subsection*{Powder enchancing equipment}

\subsubsection*{A. Fire Extinguishing Balls}
Several modern ball or grenade-style extinguishers are available in the market. They 
are manually operated by rolling or throwing into the fire. In modern version, the ball 
will explode once it comes in contact with the flame, dispersing a cloud of ABC dry 
chemical powder over the fire which extinguishes the flame. The coverage area is 
about 5 $m^2$ (54 sq. ft.).The benefit of this method is that it may be used for passive 
suppression. The ball can be placed in a fire prone area and will deploy automatically 
if a fire develops. Most modern extinguishers of this type are designed to make a loud 
noise upon deployment.

\subsubsection*{B. Condensed Aerosol Fire Suppression}
 Condensed aerosol fire suppression is a particle-based form of fire extinction 
similar to gaseous fire suppression or dry chemical fire extinction. As in the case of
gaseous fire suppressants, condensed aerosol suppressants also use clean agents to 
suppress fire. The agent can be delivered by means of mechanical operation, electric 
operation, or combined electro-mechanical operation. The difference between gaseous 
suppressants and dry chemical extinguisher is that the former emits only gas, while the 
later release powder-like particles of a large size (25-150 micrometer).The condensed aerosols 
are defined by the National Fire Protection Association as they release finely divided 
solid particles (generally <10 micrometer) in addition to gas.
The dry chemical systems must be directly aimed at the flame. The condensed 
aerosols are flooding agents and therefore are effective regardless of the location and 
height of the fire. 

\subsection*{Other Innovative Fire Extinguishers}

\subsubsection*{Water Mist Systems (Micro Fog)}
It converts liquid water to uniform fine water mist (50 to 200 micro m). It is a new 
fire extinguishing system which is clean and has enhanced fire extinguishing 
capability, marked by both high cooling capability of sprinkler system and 
excellent fire extinguishing capability of gas type fire extinguishing system. It 
suppresses fire with less water discharge even for those objects for which the 
conventional water-based fire extinguishing equipment is not viable considering
performance and economy.

\subsubsection*{Low Frequency Sound}
The basic concept behind this technology is the displacement of air by sound 
waves. Sound waves (pressure waves) displaces oxygen as they travel through the 
air. By producing sound of 30 to 60 Hz we can deprive oxygen from the space 
which inhibits the growth of fire. This technology is free from chemical and water 
and offers a relatively non-destructive method of fire control, which could find 
applications in fighting home and small scale fire.

\end{multicols}


\deco{10pt}{$\clubsuit$~$\clubsuit$~$\clubsuit$}

  \hypertarget{desalination}{\heading{Water harvesting via desalination}{6pt}}  % \hypertarget provides a label to reference using \hyperlink{label}{link text}
  { \small  \textbf{By: Shantanu Mall, Third Mechanical}\\ }

\begin{multicols}{2} % Two-column layout
Less than 3\% of the Earth's water is fresh, with most of it trapped underground or 
in ice and glaciers. That only leaves less than 1\% accessible for drinking and 
supporting life as we know it. Fresh water is finite yet we continue to waste it not 
knowing that we are dooming ourselves to a catastrophic crisis. There is an urgent 
need of new water harvesting methods and desalination is one of the most 
prominent method being used.

Desalination is a process that extracts mineral components from saline 
water. Saltwater is desalinated to produce water suitable for human 
consumption or irrigation. Feed water sources may include brackish, seawater, 
wells, surface (rivers and streams), wastewater, and industrial feed and process 
waters. Seawater desalination has the potential to reliably produce enough potable 
water to support large populations located near the coast.

The problem is that the desalination of water requires a lot of energy. Salt dissolves 
very easily in water, forming strong chemical bonds, and those bonds are difficult 
to break. The International Desalination Association says that there are about 
19,000 desalination plants operating around the world. They pumped out 
approximately 14.7 billion gallons (55.6 billion liters) of drinkable freshwater a 
day. A lot of these plants are in countries like Saudi Arabia, where energy from oil 
is cheap but water is scarce.

\subsection*{Desalination Processes}
The two basic methods for the desalination of seawater from a chemical
engineering perspective are:

\begin{description}
\item \textbf{Thermal Distillation and Membrane separation} 
  
  Thermal distillation involves heating and Boiling water turns it into vapour
 leaving the salt behind that is collected and condensed back into water by 
 cooling it down.

 \item \textbf{Reverse Osmosis}

   The leading process for desalination in terms of installed capacity and yearly 
growth is reverse osmosis (RO). The RO membrane processes use semipermeable 
membranes and applied pressure (on the membrane feed side) to preferentially 
induce water permeation through the membrane while rejecting salts. Reverse 
osmosis plant membrane systems typically use less energy than thermal 
desalination processes. Water shoots into the cylinders at a pressure of 70 
atmospheres and is pushed through the membranes, while the remaining brine is 
returned to the sea. 
\end{description}

Desalination processes are driven by either thermal (e.g., distillation) or electrical 
(e.g., RO) as the primary energy types. Energy cost in desalination processes varies 
considerably depending on water salinity, plant size and process type. RO 
membranes do not have distinct pores that traverse the membrane and lie at one 
extreme of commercially available membranes. The polymer material of RO 
membranes forms a layered, web-like structure, and water must follow a tortuous 
pathway through the membrane to reach the permeate side. RO membranes can 
reject the smallest contaminants, monovalent ions, while other membranes, 
including nanofiltration (NF), ultrafiltration (UF), and micro-filtration (MF), are 
designed to remove materials of increasing size.

\subsection*{Case Study: Israel}

Israel has proven itself as a world leader in desalination after decades of research 
and entrepreneurship. The new plant in Israel, called Sorek, was finished in late 
2013 but is just now ramping up to its full capacity; it will produce 627,000 cubic 
meters of water daily, providing evidence that such large desalination facilities are 
practical. Indeed, desalinated seawater is now a mainstay of the Israeli water 
supply. Whereas in 2004 the country relied entirely on groundwater and rain, it 
now has four seawater desalination plants running; Sorek is the largest. Those 
plants account for 40 percent of Israel's water supply. Israel now gets 55 percent of 
its domestic water from desalination, and that has helped to turn one of the world's 
driest countries into the unlikeliest of water giants. Desalination is seen by some as 
a magic bullet, the shield that saved Israel from the whims of nature.

Need of the hour-
Despite the economic and environmental hurdles, desalination is becoming 
increasingly attractive as we run out of water from other sources. We are 
overpumping groundwater, we have already built more dams than we can afford 
economically and environmentally, and we have tapped nearly all of the accessible 
rivers.


The world faces an unprecedented crisis in water resources management, 
with profound implications for global food security, protection of human health, 
and maintenance of all ecosystems on Earth. Large uncertainties still plague 
quantitative assessments of climate change impacts and water resource 
management
Fortunately, the human race has a reputation for having the irrepressible ability to 
adapt in times of great adversity by rising to meet great challenges. We need to 
realise that it is not too late to make the necessary changes and that even in our 
own personal capacity, we have the ability to change the world.

\begin{quote}
\textsl{``Save Water, Save Life''} --- \textrm{Anonymous}
\end{quote}


\end{multicols}

\deco{10pt}{$\clubsuit$~$\clubsuit$~$\clubsuit$}

\end{document}
